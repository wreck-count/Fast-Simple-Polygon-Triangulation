C++ Implementation polygon triangulation algorithm making use of \hyperlink{structDCEL}{D\+C\+EL}; Time Complexity O(nlogn); Done as part of the Computational Geometry Course at B\+I\+TS Pilani -\/ Hyderabad, under Prof. Tathagata Ray.

\subsection*{Instructions}

{\ttfamily cd src}

{\ttfamily make clean}

{\ttfamily make}

{\ttfamily ./triangulate}

\subsection*{T\+O-\/\+DO\+:}


\begin{DoxyItemize}
\item \mbox{[}X\mbox{]} \hyperlink{structDCEL}{D\+C\+EL}
\item \mbox{[}X\mbox{]} Make Monotone
\item \mbox{[}X\mbox{]} Triangulate
\item \mbox{[}X\mbox{]} Performance Analysis
\item \mbox{[}X\mbox{]} Stress test with large cases (Important)
\item \mbox{[}X\mbox{]} Documentation, Comments, A\+PI Interface, Directory Structure
\end{DoxyItemize}

\subsection*{Support}

Contact Harivallabha Rangarajan or Sathyaram, for any questions regarding running the code.

\subsection*{Authors}


\begin{DoxyItemize}
\item Sathyaram, Department of Chemistry and Computer Science, B\+I\+TS Pilani -\/ Hyderabad.
\item Harivallabha Rangarajan, Department of Mathematics and Computer Science, B\+I\+TS Pilani-\/\+Hyderabad.
\end{DoxyItemize}

\subsection*{Acknowledgements}

We acknowledge Geogebra, and Desmos for helping us visualize. Thank you! 